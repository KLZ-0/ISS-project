\documentclass[a4paper, 11pt]{article}

\usepackage[utf8]{inputenc}
\RequirePackage[left=2cm,text={17cm, 24cm},top=3cm]{geometry}
\usepackage{times}
\usepackage{graphicx}
\usepackage{setspace}
\usepackage{hyperref}
\usepackage{fancyhdr}

% for indented itemize list
\usepackage{enumitem}

\graphicspath{{../src/outputs/}}

\begin{document}

	% header
	\begin{center}
	{\includegraphics[width=0.4\linewidth]{fig/logo_en.png}}
		\\
		[0.4cm]

		{\LARGE
		ISS - Project \\
		[0.4cm]
		}

		{\large
		Adrián Kálazi (xkalaz00) \\
		\today
		}
	\end{center}

	% actual content
	\section*{Solution}

	The calculations were implemented using \textbf{Python (3.5+)} with the following libraries:

	\begin{itemize}
		\item numpy - arrays, \texttt{fft}, \texttt{ifft}
		\item scipy - \texttt{lfilter}
		\item soundfile - wav reading/writing
		\item matplotlib - plotting
	\end{itemize}

	The implementation consists of a Python module and a launcher script (\texttt{main.py}).

	The script can be run by executing \texttt{python3 ./main.py} in the \texttt{src/} folder.

	\begin{enumerate}
		\item
		The audio was recorded with \texttt{\href{https://apps.kde.org/en/kwave}{KWave}} in 16-bit resolution at 48~kHz.

		The sampling rate was later scaled down to 16~kHz using \texttt{ffmpeg}

		Durations of recorded signals: \\
		\begin{tabular}{ | l | l | l | }
			\hline
			\textbf{File} & \textbf{Samples} & \textbf{Seconds} \\ \hline
			maskoff\_tone.wav & 24806            & 1.55             \\ \hline
			maskon\_tone.wav  & 17137            & 1.07             \\ \hline
		\end{tabular}

		\item
		The sentence recording process was the same as above

		Durations of recorded signals: \\
		\begin{tabular}{ | l | l | l | }
			\hline
			\textbf{File}     & \textbf{Samples} & \textbf{Seconds} \\ \hline
			maskoff\_sentence.wav & 41162            & 2.57             \\ \hline
			maskon\_sentence.wav  & 43312            & 2.71             \\ \hline
		\end{tabular}

		\item
		Frame duration $t_F = 20\,ms$ \\
		Sampling frequency $F_s = 16\,kHz$

		Frame size in samples: $n_F = t_F \times F_s = 20~ms \times 16~kHz = 320~samples$

		\includegraphics[width=\linewidth]{3_frames.pdf}

		\item
		Variance and mean of base frequencies: \\
		\begin{tabular}{ | l | l | l | }
			\hline
			\textbf{Signal}   & \textbf{Mean} & \textbf{Variance} \\ \hline
			maskoff\_sentence.wav & 139.466       & 0.803             \\ \hline
			maskon\_sentence.wav  & 143.478       & 1.883             \\ \hline
		\end{tabular}

		\includegraphics[width=\linewidth]{4_frame.pdf} \\
		\includegraphics[width=\linewidth]{4_frame_clipped.pdf} \\
		\includegraphics[width=\linewidth]{4_frame_autocorrelated.pdf} \\
		\includegraphics[width=\linewidth]{4_base_frequencies.pdf}

%			The mentioned problem can be solved by
		There are two solutions for the mentioned problem that come to my mind:
		\begin{enumerate}
			\item
			Autocorrelation with higher resolution - this would eliminate the rapid changes in $f_0$
			but would require a longer computation time.
			\item
			In our solution, lag was defined as \texttt{np.argmax(autocorrelation\_array)} and therefore it was an integer.

			An alternate solution would be to approximate lag as a weighted arithmetic mean from the surrounding indices and their values
			which would make it a floating point number resulting in better precision.
			This method would eliminate rapid changes of $f_0$ in most cases.

			Compared to (a) this would be faster to compute but result in worse precision.
		\end{enumerate}
	\end{enumerate}

\end{document}
